\section{
    In an object-oriented inheritance hierarchy, each level is a more
    specialized form of the preceding level. Given an example of a hierarchy
    found in everyday life that has this property. Some types of hierarchy
    found in everyday life are not inheritance hierarchies. Given an example
    of a non inheritance hierarchy.
}

There are quite a few types of hierarchies I found which are not inheritance
hierarchy.

One of them is a hierarchy in offices. Branch manager do not inherit all or any
type of actions from a General manager, yet they are considered in a hierarchy
of power and level of decision making.

Another common hierarchy I found is collection of some areas is a city.
Collection of city is one district. Here, one district can have many cities
under it, but they do not inherit any kind of behaviours from district. Similar
kind of observation we can have for collection of areas of one city. District
and areas are related with each other via city.

\section{
  Look up the definition of \textit{paradigm} in at least three dictionaries.
  Relate these definitions to computer programming languages.
}

  Meaning of Paradigm in Oxford Learner's Dictionary
  \cite{paradigm_oxford_learners_dictionary} is ``a typical example or pattern
  of something```.

  Meaning of Paradigm in Collins Dictionary \cite{paradigm_collings_dictionary}
  when it is a variable noun is ``` A paradigm is a model for something which
  explains it or shows how it can be produced.``` Another meaning is when it is
  countable noun ```A paradigm  is a clear and typical example of
  something.```.

  Meaning of Paradigm in Cambridge Dictionary
  \cite{paradigm_cambridge_dictionary} when considered as noun is ```a set of
  theories that explain the way a particular subject is understood at a
  particular time.``` Another meaning is ```A model of something, or a very
  clear and typical example of something.```

  After reading meaning of paradigm explained by three different dictionaries,
  when I compare them with a computer programming language I can figure out
  that a paradigm is a way to model solutions using programming language. This
  way to craft solutions can be applied to any programming language belonging
  to that paradigm.

\section{
  Take a real world problem, such as the task of sending flowers in our
  example, and describe its solution in terms of agents(objects) and
  responsibilities.
}

  My fiance is living too far from me. I will describe how many agents are
  involved if I have to send flowers to her.

  If I purchase flowers from my town and then send them via any transportation
  service then it will nearly take 3-4 days. Flowers are delicate. So this
  option is not workable.

  Another option is to find flower suppliers from her town. I will search for
  flower suppliers who are located at her town via any search engine services
  like Google or DuckDuckGo. Here both of them are an agents who are
  responsible for giving a list of florists available in whatever location I
  give to them. I am least concern with what steps any of them are performing
  to find number of florist from given location. Once I get a list I inquire
  each florist for their rates to deliver on specific address. All florist have
  their individual rates to prepare and transfer flowers at location I am
  expecting to. Here, I can observe that each florist will have a way to
  calculate rates to deliver flowers at dedicated location, but their way to
  calculate rates are different from each other. Once I have finalised a best
  one from each of them, I give a responsibility to deliver a flower at
  specific time to a location I inquired before by paying money to that
  florist. Now, its a responsibility of florist to deliver a flower to location
  I have given. He can deliver on truck or scooter, I am least concern with it.
  I am skipping my interaction with a payment gateway which is responsible to
  take money from me and transfer it to florist.

\section{
  If you are familiar with two or more distinct computer programming languages,
  give an example of a problem showing how one language would direct the
  programmer to one type of solution and a different language would encourage
  an alternative solution.
}


  I can't remember of noticing anything good, but I will share the one which I
  have observed recently. While writing solutions of competitive programming
  problems, solutions written using the C programming language were mostly
  forwarded to use pre-allocated sequential data structure like Array. When I
  compare them with solutions written using Python, I found I can reduce my
  worry of allocating and managing dedicated storage on run-time.

\section{
  If you are familiar with two or more distinct natural languages, describe a
  situation that illustrates how one language directs the speaker in a certain
  direction and the other language encourages a different line of thought.
}

  Skipping this. Will answer later.

\section{
  Argue either of or against the position that computing is basically
  simulation. (You may want to read Kay's 1977 Scientific American article).
}

  Computing or in modern world Programming is a simulation because in
  simulation one defines a set of rules for set of entities first. And then
  each entity behaves as per their rules according to events or situations to
  simulate real-world. State of each of them is adjusted by their defined
  actions and rules. Their state is not pre-defined, but it is updated via
  actions performed on run time. Comparing this process to programming, rules
  and set of actions of each entity is predefined and then actions change a
  state of object. State can be anything on runtime.

\begin{thebibliography}{9}

  \bibitem{paradigm_oxford_learners_dictionary}
    Paradigm Oxford Learners Dictionary
    \\\textit{
      https://www.oxfordlearnersdictionaries.com/definition/english/paradigm?q=paradigm
    }

  \bibitem{paradigm_collins_dictionary}
    Paradigm Collins Dictionary
    \\\textit{
      https://www.collinsdictionary.com/dictionary/english/paradigm
    }

  \bibitem{paradigm_cambridge_dictionary}
    Paradigm Cambridge Dictionary
    \\\textit{
      https://dictionary.cambridge.org/dictionary/english/paradigm
    }
  \end{thebibliography}
